%----------------------------------------------------------------------------------------
%	CONFIGURATIONS
%----------------------------------------------------------------------------------------

\documentclass[12pt,a4paper,oneside]{article}

\usepackage[utf8]{inputenc}
\usepackage{graphicx}
\usepackage{epstopdf}
\usepackage{natbib}
\usepackage{amsmath}
\usepackage{lipsum}
\usepackage{caption}
\usepackage{subcaption}
\usepackage[a4paper,left=2cm,right=2cm,top=2.5cm,bottom=2.5cm]{geometry}

%----------------------------------------------------------------------------------------
%	INFORMATION
%----------------------------------------------------------------------------------------

\title{Métodos de pesquisa no jogo de \textit{Sesqui}\\
  \vspace{0.1in}
  \large{Inteligência Artificial - Trabalho 2}
}

\author{Miguel Ferreira\footnote{Miguel Ferreira - 201304200}, Pedro Paredes\footnote{Pedro Paredes - 201205725}, DCC - FCUP}

\date{Abril 2015}

\renewcommand{\tablename}{Tabela}
\renewcommand{\figurename}{Figura}
\renewcommand{\refname}{Referências}

\begin{document}

\maketitle

%----------------------------------------------------------------------------------------
%	SECTION 1
%----------------------------------------------------------------------------------------

\section{Introdução}
\label{sec:intro}

A existência de jogos surge naturalmente em muitas situações do
quotidiano. Em economia uma interação entre duas ou mais entidades
competidoras é normalmente vista como um jogo onde cada uma tenta
maximizar o seu lucro ou proveito. Porém, qualquer situação onde haja
cooperação ou conflito pode ser modelada como um jogo e as técnicas a
aplicar são semelhantes. Pode-se dizer assim que o estudo de jogos é o
estudo de processos de decisão.

Neste relatório exploraremos um subconjunto muito específico de jogos,
nomeadamente jogos determinísticos, discretos, de informação completa
e \textit{zero-sum} (embora alguns dos métodos se apliquem a muitos
outros tipos de jogos). A última característica é talvez a mais
importante e o que diz, informalmente, é que ``a sorte de uns é a
desgraça de outros'', o que significa que os recursos do jogo são
contantes, ou seja, o ganho de um jogador é exatamente a perda dos
seus oponentes. Mais especificamente explorar-se-á o jogo
\textit{Sesqui}, que é um jogo de tabuleiro de 2 jogadores e que será
descrito com maior pormenor numa secção futura.

O objetivo do relatório é de estudar métodos que permitam jogar
competitivamente jogos como os descritos a cima. Sendo assim
exploram-se 3 métodos diferentes: o algoritmo \textit{Minimax}, o
algoritmo de \textit{Alpha-Beta Pruning} e a \textit{Monte Carlo Tree
  Search}.

O resto do relatório está organizado da seguinte forma. Na Secção
\ref{sec:algconc} apresentamos uma descrição teórica dos 3 métodos de
otimizações comuns a cada um. Na Secção \ref{sec:sesqui} apresenta-se
o jogo \textit{Sesqui}, alguns resultados relacionados assim como uma
motivação para a implementação. Na Secção \ref{sec:imp} descreve-se a
nossa implementação dos métodos da Secção \ref{sec:algconc} e como
foram usados, assim como as heurísticas aplicadas ao
\textit{Sesqui}. Na Secção \ref{sec:resdes} discutem-se os resultados
e o desempenho da implementação. Finalmente, na Secção \ref{sec:conc}
é feita uma nota final sobre o trabalho e o relatório.

%----------------------------------------------------------------------------------------
%	SECTION 2
%----------------------------------------------------------------------------------------

\section{Algoritmos e Conceitos de Pesquisa em Jogos \textit{zero-sum}}
\label{sec:algconc}

\cite{Dutra:2015}
\lipsum[1]

\lipsum[2]

\lipsum[3]

\subsection{O Teorema e o Algoritmo Minimax}

\lipsum[1]

\lipsum[2]

\lipsum[3]

\lipsum[4]

\subsubsection{O Negamax}

\lipsum[1]

\lipsum[2]

\subsection{O Método \textit{Alpha-Beta Pruning}}

\lipsum[1]

\lipsum[2]

\lipsum[3]

\lipsum[4]

\lipsum[5]

\lipsum[6]

\subsubsection{Otimizações Comuns}

%(http://chessprogramming.wikispaces.com/Alpha-Beta#Enhancements)
%(http://www.arimaa.com/arimaa/papers/ThomasJakl/bc-thesis.pdf Section 3.1)

Tópicos: Transposition Table, Iterative Deepening, Aspiration Windows, Move ordering, Quiescence Search
Referir: NegaScout

\lipsum[1]

\lipsum[2]

\lipsum[3]

\lipsum[4]

\lipsum[5]

\subsection{\textit{Monte Carlo Tree Search}}

\lipsum[1]

\lipsum[2]

\lipsum[3]

\subsubsection{\textit{Upper Confidence Bounds} e o \textit{Multi-Armed Bandit}}

\lipsum[1]

\lipsum[2]

\subsubsection{Heurísticas: \textit{Heavy} e \textit{Light} \textit{Playouts}}

\lipsum[1]

\lipsum[2]

\subsubsection{Otimizações Comuns}

%(http://www.arimaa.com/arimaa/papers/ThomasJakl/bc-thesis.pdf Section 3.2)
%(http://arimaa.com/arimaa/papers/TomasKozelekThesis/mt.pdf Section 3.3)

Tópicos: Transposition Tables, Progressive bias, History heuristics, Best-of-N, Virtual Visits, Children Caching, Maturity Threshold
Referir: RAVE

\lipsum[1]

\lipsum[2]

\lipsum[3]

\lipsum[4]

\lipsum[5]

%----------------------------------------------------------------------------------------
%	SECTION 3
%----------------------------------------------------------------------------------------


\section{O Jogo de Sesqui}
\label{sec:sesqui}

(Porque escolhemos o jogo)

\lipsum[1]

\subsection{Regras do Jogo}

\lipsum[1]

\lipsum[2]

\lipsum[3]

\subsection{Alguns Resultados sobre o Jogo}

(Numero de tabuleiros, empates, vantagem do segundo)

\lipsum[1]

\lipsum[2]

\subsection{Intuição para a Implementação}

\lipsum[1]

\lipsum[2]

%----------------------------------------------------------------------------------------
%	SECTION 4
%----------------------------------------------------------------------------------------

\section{Implementação}
\label{sec:imp}

\lipsum[1]

\subsection{Minimax}

\lipsum[1]

\lipsum[2]

\subsection{\textit{MCTS}}

\lipsum[1]

\lipsum[2]

\lipsum[3]

\subsection{Heurísticas}

\lipsum[1]

\lipsum[2]

\lipsum[3]

%----------------------------------------------------------------------------------------
%	SECTION 5
%----------------------------------------------------------------------------------------

\section{Resultados e Desempenho}
\label{sec:resdes}

\lipsum[1]

\lipsum[2]

\lipsum[3]

%----------------------------------------------------------------------------------------
%	SECTION 6
%----------------------------------------------------------------------------------------

\section{Conclusão e Trabalho Futuro}
\label{sec:conc}

Paralelismo, Otimizações no AB, MC-AB (juntar os dois)

\lipsum[1]

\lipsum[2]

\bibliographystyle{plain}
\bibliography{lab_report_2}

\end{document}
